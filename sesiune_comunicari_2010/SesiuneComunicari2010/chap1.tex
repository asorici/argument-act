% ---- introduction, motivation, state of art, what we bring new, summary  ~ 2 pages ----
\section{Introduction}
\par
During the last decade Argumentation has been gaining importance within Artificial Intelligence especially in multi agent systems. Argumentation based dialogues have proved to be a general approach to agent communication in which the agents exchange not only statements of what they believe and what they want but also the reasons why. There is some research made in the field and several abstract argumentation frameworks were proposed like \cite{amgoud2000modelling}, \cite{parsons1998agents}.
\par
A nice overview on argumentation tasks and concepts was given by Douglas Walton in his research reports, especially in the first chapter of \cite{rahwan2009argumentation}. He distinguishes four challenges undertaken by argumentation: identification of arguments which means finding the premises and conclusion of arguments in a text and fitting that argument in an argumentation scheme; analysis which involves the identification of implicit premises or conclusion of arguments (such an argument is called an enthymeme); evaluation where the strength of an argument must be determined using a general criteria; invention which means constructing new arguments that can be used to prove some conclusion.
\par
The general approach or methodology of argumentation can be described as distinctively different from the traditional approach based on deductive logic. The traditional approach concentrated on a single inference, where the premises and conclusion are designated in advance, and applied formal models like propositional calculus and quantification theory determine whether the conclusion conclusively follows from the premises. This approach is often called monological. In contrast, the argumentation approach is called dialogical (or dialectical) in that it looks at two sides of an argument, the pro and the contra. According to this approach, the method of evaluation is to examine how the strongest arguments for and against a particular proposition at issue interact with each other, and in particular how each argument is subject to probing critical questioning that reveals doubts about it. By this dialog process of pitting the one argument against the other, the weaknesses in each argument are revealed, and it is shown which of the two arguments is the stronger.
\par
Our goal is to build an argumentation based framework for multi-agent interaction. Agents will use a public communication protocol to exchange arguments but will also use natural language to engage into dialogues with humans. Several other multi-agent argumentation frameworks were proposed for resolution of conflicts amongst desires and norms \cite{modgil2009argumentation}, practical reasoning \cite{amgoud2009constrained}, ontology mapping \cite{trojahn2009argumentation} or information exchange in prediction markets \cite{ontañón2009argumentation}. As it will be shown, in our system agents can have different topic ontologies, providing a more general framework for argumentation which is not restricted to a given subject.
\par
For the natural language processing module our approach uses statistical methods for extracting the argumentative clauses, separate arguments in text and integrating sentences into argumentation schemes.
\par
Some relevant work in this area has been done by R.M. Palau and M-F Moens in \cite{Palau}. In the paper they analyze the main research questions when dealing with argumentation mining and the different methods they have studied and developed in order to successfully confront the challenges of argumentation mining in legal texts.
\par
A more structured approach is taken by Safia ABBAS , Hajime SAWAMURA in \cite{abbas-ales}. The environment uses different mining techniques to manage a highly structured arguments repository.
\par
We use AraucariaDB corpus to train our classifiers and to compare our results to those obtained by Palau and Moens. AraucariaDB is an online database of marked up arguments maintained at the University of Dundee. The argument markup language (AML) \cite{reed2004araucaria} defines a set of tags that indicate delimitation of argument components, tags that indicate support relationships between those components, and tags that indicate the extent of instances of argumentation schemes.
