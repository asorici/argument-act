% This is lnbip.tex the demonstration file of the LaTeX macro package for
% Lecture Notes in Business Information Processing from Springer-Verlag.
% It serves as a template for authors as well.
% version 1.0 for LaTeX2e
%
\documentclass[lnbip]{svmultln}
%
\usepackage{makeidx}  % allows for indexgeneration
% \makeindex          % be prepared for an author index
%
\usepackage{graphics}
\usepackage{graphicx}
\begin{document}
%
\mainmatter              % start of the contribution
%
\title{An Multi Agent System for Argumentation}
\subtitle{Research Report}
%
\titlerunning{Argumentation MAS}  
% abbreviated title (for running head)
% also used for the TOC unless
% \toctitle is used
%
\author{Alexandru Sorici \and Alin Danciu \and Tudor Berariu}
%
\authorrunning{A. Sorici, A. Danciu, T. Berariu}
% abbreviated author list (for running head)
%
%%%% list of authors for the TOC (use if author list has to be modified)
\tocauthor{Alexandru Sorici, Alin Danciu, Tudor Berariu}
%
\institute{Faculty of Automatic Control and Computer Science, \\ University "Politehnica" of Bucharest, Romania}

\maketitle              % typeset the title of the contribution

\begin{abstract}        % give a summary of your paper
In this research report we present a framework for argumentation in multi agent systems with natural language features that supports interaction with human agents. The system uses the AIF ontology and each agent may specialize in one or more domains using one or more domain specific ontologies.
% please supply keywords within your abstract
\keywords {natural language processing, machine learning, argumentation, statistical learning, multi agent system}
\end{abstract}
%
% ---- introduction, motivation, state of art, what we bring new, summary  ~ 2 pages ----
\section{Introduction}
\par
During the last decade Argumentation has been gaining importance within Artificial Intelligence especially in multi agent systems. Argumentation based dialogues have proved to be a general approach to agent communication in which the agents exchange not only statements of what they believe and what they want but also the reasons why. There is some research made in the field and several abstract argumentation frameworks were proposed like \cite{amgoud2000modelling}, \cite{parsons1998agents}.
\par
A nice overview on argumentation tasks and concepts was given by Douglas Walton in his research reports, especially in the first chapter of \cite{rahwan2009argumentation}. He distinguishes four challenges undertaken by argumentation: identification of arguments which means finding the premises and conclusion of arguments in a text and fitting that argument in an argumentation scheme; analysis which involves the identification of implicit premises or conclusion of arguments (such an argument is called an enthymeme); evaluation where the strength of an argument must be determined using a general criteria; invention which means constructing new arguments that can be used to prove some conclusion.
\par
The general approach or methodology of argumentation can be described as distinctively different from the traditional approach based on deductive logic. The traditional approach concentrated on a single inference, where the premises and conclusion are designated in advance, and applied formal models like propositional calculus and quantification theory determine whether the conclusion conclusively follows from the premises. This approach is often called monological. In contrast, the argumentation approach is called dialogical (or dialectical) in that it looks at two sides of an argument, the pro and the contra. According to this approach, the method of evaluation is to examine how the strongest arguments for and against a particular proposition at issue interact with each other, and in particular how each argument is subject to probing critical questioning that reveals doubts about it. By this dialog process of pitting the one argument against the other, the weaknesses in each argument are revealed, and it is shown which of the two arguments is the stronger.
\par
Our goal is to build an argumentation based framework for multi-agent interaction. Agents will use a public communication protocol to exchange arguments but will also use natural language to engage into dialogues with humans. Several other multi-agent argumentation frameworks were proposed for resolution of conflicts amongst desires and norms \cite{modgil2009argumentation}, practical reasoning \cite{amgoud2009constrained}, ontology mapping \cite{trojahn2009argumentation} or information exchange in prediction markets \cite{ontañón2009argumentation}. As it will be shown, in our system agents can have different topic ontologies, providing a more general framework for argumentation which is not restricted to a given subject.
\par
For the natural language processing module our approach uses statistical methods for extracting the argumentative clauses, separate arguments in text and integrating sentences into argumentation schemes.
\par
Some relevant work in this area has been done by R.M. Palau and M-F Moens in \cite{Palau}. In the paper they analyze the main research questions when dealing with argumentation mining and the different methods they have studied and developed in order to successfully confront the challenges of argumentation mining in legal texts.
\par
A more structured approach is taken by Safia ABBAS , Hajime SAWAMURA in \cite{abbas-ales}. The environment uses different mining techniques to manage a highly structured arguments repository.
\par
We use AraucariaDB corpus to train our classifiers and to compare our results to those obtained by Palau and Moens. AraucariaDB is an online database of marked up arguments maintained at the University of Dundee. The argument markup language (AML) \cite{reed2004araucaria} defines a set of tags that indicate delimitation of argument components, tags that indicate support relationships between those components, and tags that indicate the extent of instances of argumentation schemes.

% ---- results and conclusions ~ 2 pages ----

\section{Results and Future Work}
\label{sec:results}

\paragraph*{}The role of the segmentation module is that to partition different argumentative sentences (units) into their corresponding arguments, that is, to determine the limits of each individual argument.
\paragraph*{}As presented in the description of the NLP module capabilities, we opt to calculate the semantic distance between the different argumentative units, and group sentences in one argument if they discuss content that is semantically related.
\paragraph*{}The module receives a list of argumentative sentences as its input. It then uses a word tokenizer to split the sentences into lists of words. Stopwords such as "the", "in", "with" etc. are removed before going to the next step.
\paragraph*{}We assume the semantic relatedness of words to be given by their semantic distances in a lexico-semantic resource such as WordNet. We use the lin similarity measure to compute the relatedness of two synsets. As each word in a sentence can have several senses, one would first have to go through a word sense disambiguation process to determine the most probable meaning. As this process is time-consuming, the current approach employs the use of the most common sense, as defined by WordNet, for each word we encounter.
\paragraph*{}Using the lin similarity we compute a word similarity matrix between every pair of words, one from each sentence.
\paragraph*{}To compute the estimated similarity between two sentences we then use the following method. Given two sentences A and B, for each word from sentence A we compute the most similar word from sentence B, as given by the previously computed matrix.

\[ b^* = arg \max_b Sim(a,b) \]

In a similar manner, for each word in sentence B we compute the most similar word from sentence A.

\[ a^* = arg \max_a Sim(a,b) \]

Afterwards the similarity between sentences A and B is given by
\[ ( \sum{a^*} + \sum{b^*}) / (2 * (\mid A\mid + \mid B \mid ) \]
The above process yields a sentence similarity matrix. This matrix is used as an input to a hierarchical clustering algorithm. We take an empirically determined cutoff distance of 0.875 to cut the resulting cluster dendrogram at the corresponding depth. This results in a list of sentence clusters, each of which contains the components of an individual argument.
\subsubsection*{Case study}
\paragraph*{}Here is a simple example. The initial text:
\paragraph*{}\emph{``It is well established that if any statement is made on the floor of the House by a Member or Minister which another Member believes to be untrue, incomplete or incorrect, it does not constitute a breach of privilege.  In order to constitute a breach of privilege or contempt of the House, it has to be proved that the statement was not only wrong or misleading but it was made deliberately to mislead the House.  A breach of privilege can arise only when the Member or the Minister makes a false statement or an incorrect statement willfully, deliberately and knowingly. On a perusal of the comments of the Ministers in the matter, I am satisfied that there has been no misleading of the House by them as alleged by the Member. I have accordingly disallowed the notice of question of privilege.  Copies of the comments of the Ministers have already been made available to Dr. Raghuvansh Prasad Singh.''}
\paragraph*{}The determined word lists:
\begin{description}
\item[L1:][`It', `well', 'established', 'statement', 'made', 'floor', 'House', 'Member', 'Minister',u'another', `Member', 'believes', 'untrue', 'incomplete', 'incorrect', 'constitute', 'breach', `privilege']
\item[L2:] ['In', 'order', 'constitute', 'breach', 'privilege', 'contempt', 'House', 'proved', 'statement', 'wrong', 'misleading', 'made', 'deliberately', 'mislead', 'House']
\item[L3:] ['A', 'breach', 'privilege', 'arise', 'Member', 'Minister', 'makes', 'false', 'statement', 'incorrect', 'statement', 'wilfully', 'deliberately', 'knowingly']
\item[L4:] ['On', 'perusal', 'comments', 'Ministers', 'matter', 'I', 'satisfied', 'misleading', 'House', 'alleged', 'Member']
\item[L5:] ['I', 'accordingly', 'disallowed', 'notice', 'question', 'privilege']
\item[L6:] ['Copies', 'comments', 'Ministers', 'already', 'made', 'available', 'Dr', 'Raghuvansh', 'Prasad', 'Singh']
\end{description}
\paragraph*{}The (simmetric) sentence similarity matrix:
\begin{center}
  \begin{tabular}{ | l | c | c | c | c | c | c | }
    \hline
	& \emph{1} & \emph{2} & \emph{3} & \emph{4} & \emph{5} & \emph{6} \\ \hline
    \emph{1} & 1.00 & 0.24 & 0.24 & 0.19 & 0.10 & 0.13  \\ \hline
	\emph{2} & 0.24 & 1.00 & 0.21 & 0.19 & 0.14 & 0.10  \\ \hline
	\emph{3} & 0.24 & 0.21 & 1.00 & 0.16 & 0.13 & 0.11  \\ \hline
	\emph{4} & 0.19 & 0.19 & 0.16 & 1.00 & 0.13 & 0.12  \\ \hline
	\emph{5} & 0.10 & 0.14 & 0.13 & 0.13 & 1.00 & 0.06  \\ \hline
	\emph{6} & 0.13 & 0.10 & 0.11 & 0.12 & 0.06 & 1.00  \\ \hline
  \end{tabular}
\end{center}
The resulting sentence clustering (only the indexes of the sentences are show):
\[ [[5], [4, 3, 1, 0, 2]] \]
We can see that sentences 1-5 belong to one argument and sentence 6, which is actually non-argumentative, this being only an illustrative example, belongs to a separate cluster.
\paragraph*{}The current method yields some promising results, but there is still room for improvement, especially in the process of determining word similarities, where we hope to be able to implement a more advanced measurement solution. In particular, the Lin measure can only compute the similarity between words which have the same part of speech (noun, verb etc). The lesk or gloss-vector measures, which we will try to implement in future developments can cross these boundaries and are not limited by is-a relations.
\subsubsection*{Future work}
The next step we have to take towards completing the argumentation mining module of the agents consists of extracting argumentation schemes from natural text. Our approach will use a SVM classifier trained on AraucariaDB corpus with features like the position of sentence in the text, the tense of main verb, the most probable argumentative category of previous and next sentences or argumentative patterns to split argumentative propositions into premises and conclusions. Those sentences will then be integrated into higher level structures using the schemes define by Walton in \cite{walton1996argumentation}.
% ---- architecture ~ 4 pages ----

\section{Society Overview}
\label{sec:society}

In this section we present the structure of the agent society and the argumentation framework used in interaction between agents. First our argumentation framework consists of a number of intelligent agents capable of exchanging arguments using the AIF\cite{AIF}. These agents can understand human language and are able to extract arguments from text in English. In addition to the AIF at least one topic ontology is needed in the framework to provide the ground knowledge accepted by all agents in the system. Note that the society is not constrained to use a specific ontology as a whole, but rather each agent needs at least one ontology. This allows agents to play different roles in the society analogous to the roles humans play in our society. For example an agent may have several ontologies describing legal concepts, while other agent may only know about animals. In this approach ontologies shared by all agents represent a limited form common sense knowledge. As we will see in section[\ref{sec:application}] the framework can be augmented with additional capabilities to retrieve data from external sources and insert it into AIF. 

Humans can also take part in a debate using English language. A human can debate on a given topic with one or more agents. As shown in section[\ref{sec:application}] each agent has it's own behaviour and different agents may accept different arguments as valid based on their knowledge and various behaviour parameters. In a debate each agent maintains an own view of the argumentation and will try at best to attack and defend arguments based on it's own position in the debate. If an agent can't determine the acceptability of an argument using the 

\section{Application Architecture}
\label{sec:application}

In this section we present the architecture of the agent. At the top level we have build an autonomous intelligent agent that has it's own behaviour module and reasoning capabilities. Moreover each agent can understand natural language and interact with humans and other agents.


\newpage
\begin{thebibliography}{5}

\bibitem{steg1}Karsten Stegmann, Armin Weinberger and Frank Fischer: Facilitating argumentative knowledge construction with computer-supported collaboration scripts. International Journal of Computer-Supported Collaborative Learning, Vol. 2, No. 4, Springer New York

\bibitem{palau1}Raquel Mochales Palau and Marie-Francine Moens: Argumentation Mining: The Detection, Classification and Structuring of Arguments in Text.  	 International Conference on Artificial Intelligence and Law, 2009.

\bibitem{palau2}Raquel Mochales Palau and Marie-Francine Moens :Study on the Structure of Argumentation in Case Law.Proceeding of the 2008 conference on Legal Knowledge and Information Systems, 2008.

\bibitem{safia}Safia ABBAS, Hajime SAWAMURA: ALES: An Innovative Argument Learning Environment, Proceedings of the 17th International Conference on Computers in Education, 2009.

\bibitem{araucaria}Chris Reed, Raquel Mochales Palau, Glenn Rowe and Marie-Francine Moens: Language Resources for Studying Argument.The International Conference on Language Resources and Evaluation, 2008.

\bibitem{echr}Raquel Mochales Palau and Aagje Ieven: Creating an argumentation corpus: do theories apply to real arguments?Proceedings of the 12th International Conference on Artificial Intelligence and Law, 2009.

\bibitem{ara1}Glenn Rowe, Fabrizio Macagno, Chris Reed and Douglas Walton: Araucaria as a Tool for Diagramming Arguments in Teaching and Studying Philosophy, 2009.

\bibitem{doug}Douglas Walton: Argumentation Theory: A Very Short Introduction. 	Argumentation in Artificial Intelligence, Springer US, 2009.

\bibitem{amgoud}Amgoud L., Maudet N. and Parsons S.: Modeling dialogues using argumentation. Proceedings. Fourth International Conference on Multi Agent Systems, 2000.

\bibitem{parsons}Parsons S., Sierra C. and Jennings N.:Agents that reason and negotiate by arguing.Journal of Logic and Computation, 1998

\bibitem{reed}Chris Reed and Doug Walton: Towards a Formal and Implemented Model of Argumentation Schemes in Agent Communication. Lecture Notes in Computer Science, Springer Berlin / Heidelberg, 2005

\bibitem{kiss}Kiss Tibor and Strunk Jan: Unsupervised Multilingual Sentence Boundary Detection.Computational Linguistics 32: 485-525,2006.

\bibitem{reed1}C. Reed and G. Rowe: Araucaria: Software for argument analysis, diagramming and representation.International Journal of AI Tools, 14(3-4):961{980, 2004}.

\bibitem{aif}Carlos Chesnevar, Jarred McGinnis, Sanjay Modgil, Iyad Rahwan, Chris Reed, Guillermo Simari, Matthew South, Gerard Vreeswijk and Steven Willmott: Towards an Argument Interchange Format.The Knowledge Engineering Review, pages 1-25, 2007.

\bibitem{dung}Phan Minh Dung: ON THE ACCEPTABILITY OF ARGUMENTS AND ITS FUNDAMENTAL ROLE IN NONMONOTONIC REASONING, LOGIC PROGRAMMING AND N-PERSONS GAMES, Asian Institute of Technology.

\bibitem{amg}L. Amgoud, C. Devred, and M.C. Lagasquie-Schiex.A constrained argumentation system for practical reasoning.Argumentation in Multi-Agent Systems, pages 37{56, 2009}.

\bibitem{amg1}L. Amgoud, N. Maudet, and S. Parsons. Modeling dialogues using argumentation.In Multi Agent Systems, 2000. Proceedings. Fourth International Conference on, pages 31{38, 2000}.

\bibitem{jiang}J. Jiang and D. Conrath.Semantic similarity based on corpus statistics and lexical taxonomy. Proceedings on International Conference on Research in Computational Linguistics, pages 19{33, 1997. Taiwan}.

\bibitem{lin}D. Lin. An information-theoretic definition of similarity. Proceedings of the International Conference on Machine Learning, Madison, August 1998.

\bibitem{mod}S. Modgil and M. Luck. Argumentation based resolution of conflicts between desires and normative goals. Argumentation in Multi-Agent Systems, pages 19{36,2009}.

\end{thebibliography}

\end{document}
