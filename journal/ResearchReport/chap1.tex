\section{Introduction}
\subsection{History of Argumentation Theory}
\par
Argumentation has its roots in Ancient Greece, where the art of rhetoric developed first. Aristotle defines the rhetorician as someone who is always able to see what is persuasive. Correspondingly, rhetoric is defined as the ability to see what is possibly persuasive in every given case.  Argumentation theorists have searched for the requirements that make an argument correct, by some appropriate standard of proof, by examining the errors of reasoning we make when we try to use arguments. These errors have been called fallacies. Until the 1950s, the approach of argumentation was based on rhetoric and logic.
\par
In 1895, George Pierce Baker writes “The Principles of Argumentation” giving the following definition: ``Argumentation is the art of producing in the mind of someone else a belief in the ideas which the speaker or writer wishes the hearer or reader to accept'' 
In the United States debating and argumentation became an important subject on universities and colleges. In the 1960s and 1970s Perelman and Toulmin were the two of the most influential writers on argumentation. Perelman tried to find a description of techniques of argumentation used by people to obtain the approval of others for their opinions. Perelman called this `new rhetoric'. Toulmin developed his theory (starting in 1950’s) in order to explain how argumentation occurs in the natural process of an everyday argument. He called his theory `the uses of argument'. 
\par
Hamblin took, as well, a radical approach to argumentation. Considering the fact that deductive logic did not seem to be enough, Hamblin proposed a new concept for describing arguments: he considered it not just as an arbitrarily designated set of propositions, but as a move one party makes in a dialog to offer premises that may be acceptable to another party who doubts the conclusion of the argument.
Hamblin and Perelman’s work announced a new field of study: informal logic. Nowadays,  research in informal logic increasingly incorporates the approaches to argumentation found in cognate disciplines and fields like Speech Communication, Rhetoric, Linguistics, Artificial Intelligence, Cognitive Psychology and Computational Modeling. Looked at from this perspective, informal logic as a discipline is an integral part of a much broader multi-disciplinary attempt to develop an ``argumentation theory'' that can account for informal reasoning.
